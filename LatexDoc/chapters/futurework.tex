\chapter{Future Work}

\section{Free Play Mode}

The Free Play Mode was supposed to be played after completing the campaign. It would be designed to test and strengthen the knowledge gained about runes, their meaning and how to use them most efficiently.\\
To achieve this, the player would fend off randomly generated waves of enemies with all runes unlocked. Before each wave starts, information about the type of attackers is displayed in form of rune symbols and/or names. If the player was shown the runes "Ehwaz" (speed) and "Kenaz" (torch) for example, he would know to expect a wave of fast enemies with fire element and could adjust his strategies accordingly.\\
This mode could easily be implemented as a two player mode as well, with one player creating the waves of attackers using markers and the other fending them off.

\section{Wearables}

For a game like this, wearable Head Mounted Displays (HMD) could enhance the player experience quite a bit. HMDs like the Oculus Rift, the HTC Vive or Google Cardboard in combination with the Unity engine are quite easy to implement. Player input on the other hand would be more difficult. \\ 
Using a simple mouse would break the immersion a lot. Using a controller would be acceptable, but the game requires the player to use their hands to place runes so the player would have to put down and pick up the controller constantly. Gesture recognition would be the optimal solution but is very difficult and time consuming. 
