\section{Encountered Problems}

\subsection{Markertracking}

Since we were creating our project in Unity we decided to use Vuforia as our Augmented Reality tracking library, but the performance was not sufficient in the beginning.\\
The tracking was not stable enough for the amount of markers we intended to use, so we had to change a few aspects of our game.\\\\
The first thing we scraped were the enemy path markers. Instead the path spawns relative to the base rune. This solved stability issues as well as problems with very efficient path patterns made by the players.
By removing this player variable we had an easier time creating a balanced campaign mode.\\\\
The second thing we improved was the markers themselves. The first hand-drawn markers did not bring the image recognition performance we needed.\\
Therefore all the markers we used in our project were generated by a script. In the center of each marker is the central rune the marker is representing and the outlines of the marker consists of the whole rune alphabet in a random order.
Because of that, we could have multiple different instances of the same rune markers without using e.g. plain numbers.\\

\subsection{Marker printing}

The first time we printed all of the close to 100 markers, we used a very simple printer and very basic printing paper.
This resulted in white vertical lines across most of the markers, which was very bad for the image tracking. 
Moreover, after using the markers a few times, they started to bend which made the tracking even worse.
After that we used an industrial printer with very thick paper and got much better results.

