\section{Encountered Problems}
\label{sec:problems}

During development we encountered multiple problems, which are described in this chapter.

\subsection{Marker tracking}

We had issues with the performance of the Vuforia tracking library in the early stages of development.
The tracking was not stable enough for the amount of markers we intended to use, so we had to change a few aspects of our game.

The first thing we scraped were enemy path markers. The original idea was to let the player decide where the enemies run. To reduce the amount of markers, we instead used the classical tower defense approach and created the enemy path for the player relative to his village.
This solved stability issues as well as problems with very efficient path patterns made by the players.
By removing this player variable we also had an easier time creating a balanced campaign mode.

The second thing we improved were the markers themselves. The first hand-drawn markers did not bring the image recognition performance we needed. That's why we decided to create the markers using a script (cf. Section~\ref{sec:implementation:marker_generation}) and let the player print them out.
By doing that, we could use multiple different instances of the same rune markers and improved the effectiveness of the tracking.

\subsection{Marker printing}

The first time we printed all of the~-- close to 100-~~ markers, we used a very simple printer and very basic printing paper.
This resulted in white vertical lines across most of the markers, which was very bad for the image tracking process. 
Moreover, after using the markers a few times, they started to bend which made the tracking even worse.
After that we used an industrial printer with very thick paper and got much better results in tracking and marker durability in general.
This might be a problem, when players have to print the markers for themselves.

\subsection{ARCore/ARKit}

ARCore and ARKit was anounced in August 2017 and promised heavy performance boosts in the markerless tracking technology. Those advances in tracking stability could have been a great benefit for the game. 
Unfortunately, the support release for phones that are not flagship level (like the Galaxy S9 /iPhone 8) is still in the making. Dynamic environmental lighting and AR plane detection
were the features we especially looked forward to.

