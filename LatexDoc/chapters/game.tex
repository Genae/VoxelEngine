\section{Game Design}
\label{sec:game_design}

This chapter covers our game design in regard to game play and the educational aspect.

\cite{rieckhoff-celts-in-germany}

\subsection{Basic Concept}
The priority of this serious game is to convey knowledge of celtic runes and culture. 
Since the raw learning of plain foreign symbols is not very exciting, our game tries to weaken the less interesting parts of learning by offering a classical Tower Defense Game.

\subsection{Runes}

Almost every rune out of the 26 letter rune alphabet has a special meaning in the game itself, very close to the traditional meaning of the rune in ancient times.\\
To name but one, the \emph{Kenaun} rune, also called \emph{Kenaz} has the traditional meaning \emph{torch} and in-game upgrades standard towers to fire towers.

\subsection{Tower Defense}

The Tower Defense concepts we are using are very basic. Towers can be manually placed and upgraded in various ways, using gold, to kill monsters that are walking on a specific path towards the players base. If a monster reaches the base alive, it does damage to the players health and if the health drops to zero the player looses and has to start over. 
The distinguishing feature of our tower defense is that the player does not earn gold by killing monsters, but by planting and harvesting different farms.

\subsection{Campaign}

Because of the sheer number of runes the player has to learn before he is able to actually play the game, we slowly introduce the runes step by step in a campaign/ tutorial mode. That way, no one is overwhelmed by too many different rune symbols, their meanings and their meanings in-game, but is able to learn in small manageable steps. \\
After playing the campaign, a free-play mode is unlocked to offer a challenge for more experienced players without playing the campaign again. \\

\subsection{History}



Manching nachempfunden

