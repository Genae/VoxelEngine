\section{Game Design}
\label{sec:game_design}

This chapter covers our game design in regard to game play and the educational aspect.

\subsection{General Concept}
\label{sec:game_design:subsec:general_concept}

Our game design evolves from the idea of a serious game utilizing augmented reality techniques available on larger mobile devices, i.e.\ tablets or phones with a display size of at least 7 inch.
The educational goal is to convey knowledge about Celtic runes and culture through an engaging and entertaining game experience.
Learning is supported by a connection between the matter and the game mechanic, so the player improves his skill in the game by mastering knowledge about runes and vice versa.

The game is designed as a classical Tower Defense (TD) game. Players build towers to defend their city from increasingly difficult hordes of enemies.
This is explained in more detail in Section~\ref{sec:game_design:subsec:tower_defense}.
We use augmented reality to give the player control over the towers and buildings with printed physical markers, our `rune markers' cf. Section~\ref{sec:game_design:subsec:runes}.
This is meant to increase the learning success by bringing the runes into focus while providing a refreshing game experience through novel control mechanics.
Knowledge about runes and Celtic history is presented gradually using a game campaign, where the player unlocks new runes and additionally information after increasingly more difficult levels. This is thoroughly explained in Section~\ref{sec:game_design:subsec:campaign}.

Finally we based the style of the game on available historical data about Celtic cities, defenses and enemies, which is covered in Section~\ref{sec:game_design:subsec:history}.

TODO: 

\subsection{Runes and Rune Markers}
\label{sec:game_design:subsec:runes}

Almost every rune out of the 26 letter rune alphabet has a special meaning in the game itself, very close to the traditional meaning of the rune in ancient times.\\
To name but one, the \emph{Kenaun} rune, also called \emph{Kenaz} has the traditional meaning \emph{torch} and in-game upgrades standard towers to fire towers.

\subsection{Tower Defense}
\label{sec:game_design:subsec:tower_defense}

The Tower Defense concepts we are using are very basic. Towers can be manually placed and upgraded in various ways, using gold, to kill monsters that are walking on a specific path towards the players base. If a monster reaches the base alive, it does damage to the players health and if the health drops to zero the player looses and has to start over. 
The distinguishing feature of our tower defense is that the player does not earn gold by killing monsters, but by planting and harvesting different farms.



\subsection{Campaign}
\label{sec:game_design:subsec:campaign}

Because of the sheer number of runes the player has to learn before he is able to actually play the game, we slowly introduce the runes step by step in a campaign/ tutorial mode. That way, no one is overwhelmed by too many different rune symbols, their meanings and their meanings in-game, but is able to learn in small manageable steps. \\
After playing the campaign, a free-play mode is unlocked to offer a challenge for more experienced players without playing the campaign again. \\

\subsection{History}
\label{sec:game_design:subsec:history}


Manching nachempfunden

