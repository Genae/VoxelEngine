\section{Game Design}
\label{sec:game_design}

This chapter covers our game design in regard to game play and the educational aspect.

\subsection{General Concept}
\label{sec:game_design:subsec:general_concept}

Our game design evolves from the idea of a serious game utilizing augmented reality techniques available on larger mobile devices, i.e.\ tablets or phones with a display size of at least 7 inch.
The educational goal is to convey knowledge about Celtic runes and culture through an engaging and entertaining game experience.
Learning is supported by a connection between the matter and the game mechanic, so the player improves his skill in the game by mastering knowledge about runes and vice versa.

The game is designed as a classical Tower Defense (TD) game. Players build towers to defend their city from increasingly difficult hordes of enemies.
This is explained in more detail in Section~\ref{sec:game_design:subsec:tower_defense}.
We use augmented reality to give the player control over the towers and buildings with printed physical markers, our `rune markers' (cf. Section~\ref{sec:game_design:subsec:runes}).
This is meant to increase the learning success by bringing the runes into focus while providing a refreshing game experience through novel control mechanics.
Knowledge about runes and Celtic history is presented gradually using a game campaign, where the player unlocks new runes and additionally information after increasingly more difficult levels. This is thoroughly explained in Section~\ref{sec:game_design:subsec:campaign}.

Finally we based the style of the game on available historical data about Celtic cities, defenses and enemies, which is covered in Section~\ref{sec:game_design:subsec:history}.

\subsection{Runes and Rune Markers}
\label{sec:game_design:subsec:runes}

As the primary goal of playing the game is to learn about Celtic runes, the runes are the core of our game experience.
Therefore we split the game into two distinct phases:
A preparation phase where the player interacts with physical representations of the runes, our rune markers, and a game phase where the player interacts with the augmented reality device.

\begin{wrapfigure}{R}{0.38\textwidth}
	\centering
	\includegraphics[width=0.35\textwidth]{figures/algiz0.jpeg}
	\caption{\label{fig:rune-marker} An Algiz rune marker for building towers}
\end{wrapfigure}

Our rune markers are printed out representations of runes that each have a unique effect in the game and need to be properly positioned by the player. Figure~\ref{fig:rune-marker} shows a rune marker for the \textit{`Algiz'} rune, which is used to build and position towers in-game.
As a player can use some runes multiple times, the rune markers have an additional unique boarder that enables the augmented reality tracking to distinguish similar runes.

As the basis for our runes and rune markers we used the 24 runes of the \textit{`Elder Futhark'}~\cite{elder-futhark}, the oldest form of the runic alphabets.
Almost\footnotemark all of the 24 runes have a special meaning in the game that closely relates to their original meaning.
For example the Algiz rune has the meanings \textit{`Elk', `Protection', `Defense'}~\cite{algiz} and is used to build towers.
Though, it must be noted here though that runes can have multiple meanings and the meaning is not always absolutely clear.
Rieckhoff and Biel point out that there is not Celtic historiography, literature or religious writings.~\cite{rieckhoff-runes} The absence of a large basis of usage, makes interpreting runes especially difficult.
\footnotetext{Unfortunately, we had to skip some runes as there was no reasonable way of mapping the rune to an in-game functionality.}

To properly play the game a player must recognize a rune and it's function in-game. This supports in combination with the coverage of most runes and a close relation to their traditional meaning the learning success of the player.


\subsection{Tower Defense}
\label{sec:game_design:subsec:tower_defense}

The Tower Defense concepts we are using are very basic. Towers can be manually placed and upgraded in various ways, using gold, to kill monsters that are walking on a specific path towards the players base. If a monster reaches the base alive, it does damage to the players health and if the health drops to zero the player looses and has to start over. 
The distinguishing feature of our tower defense is that the player does not earn gold by killing monsters, but by planting and harvesting different farms.



\subsection{Campaign}
\label{sec:game_design:subsec:campaign}

Because of the sheer number of runes the player has to learn before he is able to actually play the game, we slowly introduce the runes step by step in a campaign/ tutorial mode. That way, no one is overwhelmed by too many different rune symbols, their meanings and their meanings in-game, but is able to learn in small manageable steps. \\
After playing the campaign, a free-play mode is unlocked to offer a challenge for more experienced players without playing the campaign again. \\

\subsection{History}
\label{sec:game_design:subsec:history}


Manching nachempfunden

