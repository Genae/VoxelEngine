\section{Introduction}
\label{sec:introduction}

\textit{``The year is 50 B.C. Gaul is entirely occupied by the Romans. Well, not entirely... One small village of indomitable Gauls still holds out against the invadors. And life is not easy for the Roman legionaries who g garrison the fortified camps of Totorum, Aquarium, Laudanum and Compendium...'', Robert Steven Caron.}~\cite{asterix-beginning}

Similar to the famous opening quote from the Asterix comics, our game is set in the middle of the last century before Christ. The Celtic culture is at its peak and shaped central Europe and the British Isles during the last centuries.~\cite{celtic-expansion} But conflict arises between the Celts and the Romans when Gaius Julius Caesar is leading his military campaigns northwards to conquer Gaul, known as the \textit{`bello Gallico'}.

During his campaigns, Caesar encounters several \textit{`oppidum'}, fortified Celtic villages or cities.~\cite{collis-oppidum} The player is taking control over the defenses of a small Celtic \textit{`oppidum'} and has to defend the village using powerful Celtic runes. After the village is fortified enough to fend of wildlife, the player has to fend of wave after wave of the Roman campaign to conquer his village. Who will come out on top? The organized roman empire or the naturalistic Celts with exceeding knowledge in ancient runes?

The game is supposed to teach players the meaning of ancient Celtic runes in a playful and engaging way. While playing the campaign players gradually unlock runes, that they must use with physical, printed markers\footnotemark \ to control the augmented reality world.
When a rune is unlocked the player gets educational information about the rune and an explanation of its effect in-game. The original meaning of the rune is related to their in-game effect to improve the learning experience by creating coherence between the educational information and the game experience.
\footnotetext{Our rune markers are explained in Chapter~\ref{sec:game_design}}

The following chapters are meant as a documentation for our prototype implementation. We start with describing the game design choices, followed by an explanation of the mechanics used to play the game. Afterwards we discuss problems during development, especially in regard to augmented reality and present a short user study of our prototype as presented at the TUM Demo Day of the summer semester 2018. Finally we discuss future work that was out of scope of the practical course.






